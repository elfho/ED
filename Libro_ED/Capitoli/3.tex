
\chapter{Valence band of Silicon} % (fold)

	\label{cha:valence_band_of_silicon}

	The valence band of silicon is made by the overlapping of three subbands:
	
	\begin{itemize}
		\item Heavy holes band
		\item Light oholes bans
		\item Spin-off band
	\end{itemize}

	These 3 subbands are all isotropic which means that they have the same effective mass in all directions:
	
	\begin{equation}
		m_x=m_y=m_z
	\end{equation}

	In detail we have:
	
	\begin{itemize}
		\item $m_{hh}=0.49m_0$ and we are able to write the dispersion relationship for the heavy holes band as: $E_v-E=\frac{\hbar^2k^2}{2m_{hh}}$
		\item $m_{lh}=0.16m_0$ and we are able to write the dispersion relationship for the light holes band as: $E_v-E=\frac{\hbar^2k^2}{2m_{lh}}$
		\item $m_{so}=0.29m_0$ and we are able to write the dispersion relationship for the spin-off band as: $E_v-E=\frac{\hbar^2k^2}{2m_{so}}$
	\end{itemize}
	
	in which $k^2=k_x^2+k_y^2+k_z^2$
	As we have isotropic bands we are able to plot E over k obtaining 3 parabolic curves, while in the $k_x$,$k_y$ and $k_z$ space we are going to have 3 spheres as larger as the effective mass is larger.

	Considering the dispersion relations we can find the intersection with axes of the spheres:
	
	\begin{equation}
		K=\pm \sqrt{\frac{2m_*(E_v-E)}{\hbar^2}}
	\end{equation}
	
	and starting from them we can find the volume of sphere which will be usefull to calculate the density of states.
	For example we can evaluate the volume of the sphere related to the heavy holes band.
	
	\begin{equation}
		V_{hh}=\frac{4}{3} \pi \left[\sqrt{\frac{2m_{hh}(E_V-E)}{\hbar^2}} \right]^3
	\end{equation}

	Let's now evaluate the number of states in the same way we have done for the conduction band of generic semiconductors in chapter~\ref{number_of_states_CB}:
	
	\begin{equation}
		N=\frac{V}{(\frac{2 \pi}{L})^3L^3}=\frac{4}{3} \pi\frac{1}{h^3} \sqrt{{8m_{hh}^3}} \sqrt{(E_V-E)^3}
	\end{equation}

	\begin{equation}
		dN=\frac{4}{3} \pi\frac{1}{h^3} \sqrt{{8m_{hh}^3}} \frac{3}{2} \sqrt{(E_V-E)}dE
	\end{equation}

	Now we are able to get the densities of states as:

	\begin{equation}
		g_v^{hh}=\frac{8 \pi}{h^3}\sqrt{2m_{hh}^3}\sqrt{E_v-E}
	\end{equation}
	
	\begin{equation}
		g_v^{lh}=\frac{8 \pi}{h^3}\sqrt{2m_{lh}^3}\sqrt{E_v-E}
	\end{equation}
	
	\begin{equation}
		g_v^{so}=\frac{8 \pi}{h^3}\sqrt{2m_{so}^3}\sqrt{E_so-E}
	\end{equation}

	from which we can find the number of carriers evaluating the number of occupied states.

	\begin{equation}
		p^{hh}=\int_{-\infty}^{E_v} g_v^{hh}(1-f(E))=\int_{-\infty}^{E_v}\frac{8 \pi}{h^3}\sqrt{2m_{hh}^3}\sqrt{E_v-E}\frac{1}{1+e^{-\frac{E-E_f}{KT}}}dE
	\end{equation}

	Equation which can be simplified as done for the conduction band usign some variable changes:
	
	\[\left[x= \frac{E_v-E}{KT} \ \ \ \ \ \ \ \ \ \ dE=-KTdx \ \ \ \ \ \ \ \ \ \ \mu=\frac{E_v-E_f}{KT} \ \ \ \ \ \ \ \ \ \ x- \mu = \frac{E_f-E}{KT}\right]\]

	\begin{equation}
		p^{hh}=\frac{8 \pi}{h^3}\sqrt{2m_{hh}^3}(KT)^{\frac{3}{2}}\frac{\sqrt{\pi}}{2} \frac{2} {\sqrt{\pi}} \int_0^{+\infty} \frac{\sqrt{x}}{1+e^{x- \mu}}dx=N_vF_{\frac{1}{2}}(\mu)
	\end{equation}
		
	we can also introduce:

	\begin{itemize}
		\item $N_v^{hh}=\frac{8 \pi}{h^3}\sqrt{2m_{hh}^3}(KT)^{\frac{3}{2}}\frac{\sqrt{\pi}}{2}$
		\item $N_v^{lh}=\frac{8 \pi}{h^3}\sqrt{2m_{lh}^3}(KT)^{\frac{3}{2}}\frac{\sqrt{\pi}}{2}$
		\item $N_v^{so}=\frac{8 \pi}{h^3}\sqrt{2m_{so}^3}(KT)^{\frac{3}{2}}\frac{\sqrt{\pi}}{2}$
	\end{itemize}

	\subsection{Holes concetration in valence band}
		we can evaluate the number of carriers considering the contribution of the three subbands:

		\begin{equation}
			p_{v_tot}=p^{hh}+p^{lh}+p^{so}
		\end{equation}

		But we know that $E_v-E_{so}=44meV$ and as $44meV>>KT$ we are able to neglect the contribution of the spin-off band.

		\begin{equation}
			p=(N_v^{hh}+N_v^{lh})F_{\frac{1}{2}}(\mu)
		\end{equation}
		Where we have $N_v^{hh}+N_v^{lh}=1.83*10^19 cm^{-3}$


	\subsection{Carriers density in Metals}
		Let's consider the case of $E_f>E_c$.
		In this case tha Maxwell-Bolzmann approximation cannot be used, but we can consider $f(E)$ as step function.

		\[n=\int_{E_c}^{E_f}g_v(E)dE=\frac{48 \pi}{h^3}\sqrt{2m_t^2m_h}\sqrt{E-E_c}dE=\left[ \frac{48 \pi}{h^3}\sqrt{2m_t^2m_h}\frac{2}{3}{E-E_c} \right]_{E_c}^{E_f}=\]
		 
		 \begin{equation}
		 = \frac{48 \pi}{h^3}\sqrt{2m_t^2m_l}(KT)^{\frac{3}{2}} \frac{\sqrt{x}}{2}	\frac{2}{\sqrt{x}} \frac{2}{3} \left(\frac{E_f-E_c}{KT}\right)^ {\frac{3}{2}} = N_c \frac{2}{3}  \frac{2}{\sqrt{x}} \left(\frac { E_f-E_c}{KT}\right)^{ \frac{3}{2} }=N_c\frac{2}{3} \frac{2}{\sqrt{\pi}} \mu^{\frac{3}{2}} 
		 \end{equation}
		
		We can compare this result with the Maxwell-Bolzman approximation ($n=N_ce^{-\frac{E_c-E_f}{KT}}=N_ce^{\mu}$) and note that it's valid only for $\mu<-3$ and that the approximation error increase with $E_f$.
