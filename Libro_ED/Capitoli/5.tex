\chapter{Electrostatic analisys of semiconductor materials} % (fold)
\label{cha:electrostatic_analisys}

Let's note how adding an exernal potential to the crystal the periodic potential configurations and the bands are shifted.
When the external potential is positive the bands are bended downwards.

Under the effective mass approximation we can consider electrons in the crystal as free particles subjected only to external forces.
So external potential is the only thing we need to invetigate the behaviour of our carriers.

\begin{equation}
 	\vec{F}=-\vec{\nabla} \phi  \ \ \ \  \Longrightarrow \ \ \ \  	\vec{F}=- \frac{d\phi}{dx}
 \end{equation} 

\begin{equation}
	\vec{\nabla}\vec{F}= \frac{\rho}{\epsilon_{si}} \ \ \ \  \Longrightarrow \ \ \ \ \frac{dF}{dx}=\frac{\rho}{\epsilon_{si}}
\end{equation}
From which we get:

\begin{equation}
	\vec{\nabla}(-\vec{\nabla}\phi)= -\vec{\nabla}^2\phi=\frac{\rho}{\epsilon_{si}}
\end{equation}

And we are able to write the \textbf{Poisson equation for semiconductor materials}
\begin{equation}
	\frac{d^2\phi}{dx^2}=-\frac{\rho^2}{\epsilon_{si}}=-\frac{q}{\epsilon_{si}}(p-n+Nd^+-Na^-)
\end{equation}

Assuming n doped silicon we can neglect holes and Na is nil, so we can write:


\begin{equation}
	\frac{d^2\phi}{dx^2}=-\frac{q}{\epsilon_{si}}(Nd^+-n)= -\frac{q}{\epsilon_{si}}(Nd + \Delta Nd(x)-n_ie^{\frac{q(\phi+\Delta \phi(x) -\phi_f)}{KT}})
\end{equation}

\begin{equation}
	\frac{d^2\phi}{dx^2}=-\frac{q}{\epsilon_{si}}\Delta Nd(x)+\frac{q^2Nd}{\epsilon_{si}KT}\Delta\phi(x)
\end{equation}
 This is a second order non omogeneous differential equation that has as result an exponential behaviour:

 \begin{equation}
 	\Delta \phi(x) \propto e^{-\frac{x}{L_d}}
 \end{equation}

 where Ld defined as the debye length is $L_d= \sqrt{\frac{\epsilon_{si}KT}{q^2Nd}}$

 
% chapter electrostatic_analisys (end)