\chapter{Doped semiconductors} % (fold)
	
	\label{cha:doped_semiconductors}
	
	In general semiconductors of the \rom{4} group are doped with elements from the \rom{3} and \rom{5} group, with a substitutional doping process.

	In a doped material we have:

	\begin{equation}
		n=p+N_d^+=N_ce^{-\frac{E_c-E_f}{KT}}=N_ve^{-\frac{E_f-E_v}{KT}}+\frac{N_d}{1+2e^{\frac{E_v-E_f}{KT}}}
	\end{equation}

	In which holes can be neglected onbtaining, and considering the material at room temperature we can assume to be in the complete ionization condition, in which $n= N_d$.\footnote{In which $N_d$ is the donors density}

	\section{Action mass law} % (fold)
		
		\label{cha:action_mass_law}

		From previous considerations we obtain the action mass law:

		\begin{equation}
			p=\frac{N_i^2}{N_d}
		\end{equation}

		But note now that increasing a lot the donors concentration we have to consider the incomplete ionization condition in which the Maxwell-Bolzman approximation is not still valid.\footnote{High doped silicon is generally called degenerated silicon}
		% chapter action_mass_law (end)

		\subsection{Relationship between Fermi level and Temperature} % (fold)
		
			\label{sub:relationship_between_fermi_level_and_temperature}

			Having in mind the~\ref{eq:Fermi} we notice that the higher is the temperature , the higher is the probability to have occupied states in the conduction band.
			But $E_f$ cannot keep decreasing; infact if we increase the temperature intrinsic carriers concentration increase and cannot be neglected, so $E_f$ asintotically approches $E_i$.
			On the other side if we decrease the temperature $E_f$ approches $E_d$ and incomplete ionization must be considered.\footnote{All these considerations can be repeated for acceptors dopants}

		% subsection relationship_between_fermi_level_and_temperature (end)

	\section{Carriers transport} % (fold)
		
		\label{cha:carriers_transport}

		\subsection{Drift} % (fold)

			\label{sub:drift}
			If we aplly an electric field it will accelerate as electrons as hole \footnote{In opposite directions} , and we will aspect a linear relationship, but carriers are moving in a crystal and they are subjected to scattering events.

			We can find for both carriers an average value of the velocity given by:

			\begin{equation}
				V_{d_n}= \mu_n F
			\end{equation}

			\begin{equation}
				V_{d_p}= \mu_p F
			\end{equation}

			Where $ \mu_n$ and $\mu_p$ are the mobilities for respectively electrons and holes.

			\begin{equation}
				J_{drift_n}=qn \mu_n F \ \ \ \ \ \ \ \ \ \ J_{drift_p}=qp \mu_p F 
			\end{equation}

		\subsection{Mobility} % (fold)
			\label{sub:mobility}
			Mobility changes has a relationship with:
			
			\begin{itemize}
			 	\item Doping concentration
			 	\item Temperature
			 	\item Dimensionality
			 \end{itemize} 
			
			 \subsubsection{Mobility and doping concentration}
			
				If we increase the doping concentration we increase the number of impurities and the probability of scattering events with them.
				There are 2 different types particles with which carriers can scatter:
			
				\begin{itemize}
					\item Phonons
					\item Impurities
			
				\end{itemize}
			\subsubsection{Mobility and temperature}
				
				If the temperature increase also the probability of scattering increase and so the mobility is smaller.
			
			\subsubsection{Mobility and dimensionality}
			
			Near surface the surface the number of impurities is much more higher and so increasing the scattering probability the mobility deacrease.

			The relationship given for the drift velocity is still linear and it seems that we are able to increase the drift velocity freely; this is not true as when we reach high levels of 
			electric field the scattering contribution given by optical phonons\footnote{High energetic particles, generated by lattice vibrations} is not still negligible.
		
		\subsection{Diffusion}
			Carriers transport is also given by the diffusion phenomena which is simply described by:
			\begin{equation}
				J_{diff_n}=qD_n\frac{dn}{dx} \ \ \ \ \ \ \ \ \ \   J_{diff_p}=qD_p\frac{dp}{dx}
			\end{equation}

			In which $D_n$ and $D_p$ are the diffusion coefficients given by the Einstein's relation:

			\begin{equation}
				D_n=\frac{KT \mu_n}{q} \ \ \ \ \ \ \ \ \ \ D_p=\frac{KT \mu_p}{q}
			\end{equation}

		\subsection{Total density of current}

			\begin{equation}
				J_n=q n\mu_nF+D_n\frac{dn}{dx} \ \ \ \ \ \ \ \ \ \ J_p=q p\mu_pF+D_p\frac{dp}{dx}
			\end{equation}
			% subsection mobility (end)
			% subsection drift (end)
			% chapter carriers_transport (end)
